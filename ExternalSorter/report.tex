\documentclass{article}
% Přepneme na českou sazbu
\usepackage[czech]{babel}
\usepackage[IL2]{fontenc}


%% Použité kódování znaků: obvykle latin2, cp1250 nebo utf8:
\usepackage[utf8]{inputenc}
\begin{document}


\section*{Načítání vstupních dat}
O načtení a rozparsování dat se stará třída $InputNumberStream$, která používá C-čkové streamy.
Vstup je čten do bufferu, kde je pak rozparsován pomocí Hornerova schematu. S řetězci je pracováno
přímo v buffer pomocí $char*$ pointerů. Rozparsovana položka se potom uloží do struktory $Entry$, která 
má 32-bitovou položku pro číslo řádko a 64 bitovou položku pro číslo. U této struktury je nastaveno, 
aby ji překladač nezarovával a měla tak 12 B. Tyto položky jsou načteny do bufferu, který reprezentuje 
třída $Chunk$. 


\section*{Třídění}
Třídění probíhá tak, že se vytvoří až 6 GB velké chunky, které jsou vnitřně stříděny a pak mergeovány 
vnějším tříděním, které reprezentuje třída ExternalSorter.  Na počátku se nalokuje 8 GB buffer, který se pak rozděluje dalším strukturám, 
tak aby nedocházelo zbytečně k alokaci a dealokaci. 6GB z tohoto bufferu je vyuzito na ulozeni 
načtených položek, zbylé 2 GB jsou pak použity jako buffery při třídění. 

O vnitřní třídění se stará 
třída $SimpleChunkCreator$, která načte data do 6 GB bufferu. Tento buffer je setříděn po 64 MB blocích,
reprezentovaných třídu $SubChunk$, pomocí hybridního třídění. 

Hybridní je třídění realizované quicksortem,
který pro délku meně než 13 prvků volá insert sort. Pro tři a dva prvky je sekevence setříděná triviálně
podmínkami. Pokud quick sort dojde do 2xkrát lograritmické hloubky, nebo není nalezen efektivní pivot
je zavolán heapsort. Pivot se volí jako průměr prvního a posledního prvku sekvence. 

Bloky jsou pak mergeovany "pohybem" bufferu tam a zpět, dokud nevzniknou tři setříděné posloupnosti, které
ty jsou slity přímo do souboru. Při mergeovaní probíhá rovněž odstraňování duplicit, tím může dojít
ke značnému zmenšní dat v buffer. Proto je proveden pokus o doplnění tohoto bufferu. Je možné donačíst
buffer o velikosti $x$ tak, aby mohl proběhnout další mergeovaní, pro které je třeba buffer. 

$$ x \le (a - 2*b)/2 $$

kde $a$ je velikost volného bufferu po odstranění duplicit a $b$ je velikost nejmenšího setříděné posloupnosti.
Načtená data velikosti $x$ jsou poté setříděna obdobný způsobem, nicméně celá do jedné sekvence. Tato sekvence
je pak spojená s nejmenší setříděnou posloupností $b$. Tím nám vzniknou opět tři setříděné sekvence, které
mohou mergeovany přímo do souboru.

Až na poslední vrstvu jsou data do souboru ukládána respektivě čtena binárně pomocí tříd $InStream$ a $OutputStream$.

\end{document}



